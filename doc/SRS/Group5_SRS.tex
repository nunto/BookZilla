\documentclass[12pt]{article}
    
    \usepackage{graphicx}
    \usepackage{paralist}
    \usepackage{amsfonts}
    \usepackage{listings}
    \usepackage{hyperref}
    \usepackage{tabto}
    \usepackage[table]{xcolor}
    \usepackage{float}
    \usepackage[normalem]{ulem}
    
    \hypersetup{colorlinks=true,
        linkcolor=blue,
        citecolor=blue,
        filecolor=blue,
        urlcolor=blue,
        unicode=false}
    
    \oddsidemargin 0mm
    \evensidemargin 0mm
    \textwidth 160mm
    \textheight 200mm
    
    \pagestyle {plain}
    \pagenumbering{arabic}
    
    \newcounter{stepnum}
    
    \usepackage{fancyhdr}
    \usepackage{fancyhdr}
    \fancyhead[L]{\today\ }
    \fancyhead[C]{SRS}
    \fancyhead[R]{Group 5}
    \pagestyle{fancy}
    
    \title{Software Requirements Specification: Revision 0}
    \author{Group 5 \\
            \\ Lu, Daniel - lud1  - 400015933
            \\ Abujarad, Razan - abujarar  - 400038238
            \\ Panunto, Michael - panuntom - 400022970
            \\ Samarasinghe, Sachin - samarya - 001430998
            \\ Bengezi, Mohamed - bengezim - 400021279 \\
            \\ Professor: Dr. Kehdri
            \\ Tutorial: T02}
    
    \begin {document}
    
    % Options to be used by the tables
    \setlength{\arrayrulewidth}{1.5pt}
    \definecolor{light-gray}{gray}{0.93}
    
    \maketitle
    
    \newpage
    
    {\centering
      \tableofcontents\par
    }
    \addtocontents{toc}{\protect\vspace{2.5em}}
    \newpage
    
    \section{Introduction}
    \label{sec:introduction}
    % Begin Section
    
    This section of the SRS should provide an overview of the entire SRS.
    
    \subsection{Purpose}
    \label{sub:purpose}
    \tab This SRS describes both the functional and non-functional requirements necessary for the system-to-be. Specifically, this document is intended for users or any parties interested in the BookZilla application. The objective is to allow potential users and stakeholders to gain insight into the main goals of the application, as well as the thought process behind it. This document aims to describe all the functionalities, necessary requirements, constraints, dependencies, and any other topics that may be of importance to the development of the application.
    
    \subsection{Scope}
    \label{sub:scope}
    \tab BookZilla is a recognition application that allows users to find books that they may have forgotten about, or simply to help them discover new books. It is designed for the Android platform. The application will take various user inputs, such as genre, theme, or description, and uses a separate expert for each one of these inputs. Using these experts, the application determines a list of books that each expert agrees on. This application has numerous benefits. For example, it has the potential to assist an individual in finally remembering the name of the marvelous book they once read some time ago. As stated above, it also could help in discovering new books, since it outputs a list of books. The goal of BookZilla is to be as accurate as possible with the given inputs, in order to fully realize these stated benefits.
    % End SubSection
    
    \subsection{Definitions, Acronyms, and Abbreviations}
    \label{sub:definitions_acronyms_and_abbreviations}
    % Begin SubSection
    \begin{table}[H]
        \begin{center}
            \rowcolors{1}{white}{light-gray}
            \begin{tabular}{| p{0.30\linewidth} | p{0.70\linewidth} |}
                \hline
                \textbf{Term} & \textbf{Definition}\\
                \hline
                API & (Application Programming interface) Set of subroutines used to assist in the completion of tasks within the application\\
                UI & (User Interface) Section of the application where users can interact with the program\\
                UI Element & Basic graphical elements that make up the UI. Text-boxes, buttons are some examples\\
                Properties of a Book & Attributes of a book that can be used to identify it. For example the author, release date, title \\ 
                \hline
            \end{tabular}
            \caption{Definitions} \label{tab:Defs}
        \end{center}
    \end{table}
    % End SubSection
    
    \subsection{References}
    \label{sub:references}
    % Begin SubSection
        Google Books \textbf{API}: 
        \begin{itemize}
            \item https://developers.google.com/books/
        \end{itemize}
    % End SubSection
    
    \subsection{Overview}
    \label{sub:overview}
    % Begin SubSection
    The remainder of the document is divided among three chapters. Chapter two outlines the general functionality of the application alongside the characteristics and constraints surrounding the application and relevant stakeholders. Additionally, future requirements and functionalities are layed out. \\ ~ \\
    Chapter three defines the technical specifications of the Android application required to perform its functionality. It consists of individual requirements and rationales to be implemented in order to achieve the intended result. \\ ~ \\
    Finally, chapter four specifies criteria used to evaluate the operation of the application based on the requirements outlined in chapter three. It outlines how the system should operate to fulfill its functionality.
    % End SubSection
    
    % End Section
    
    \section{Overall Description}
    \label{sec:overall_description}
    % Begin Section
    
    
    \subsection{Product Perspective}
    \label{sub:product_perspective}
    % Begin SubSection
    
    BookZilla is a mobile application designed to make the process of searching for a book straightforward and simple. BookZilla can use different properties of a book to discover similar or related books. The software is not self contained as it requires access to the Google Books API to retrieve books based off of search terms. However this product is not a product that is a component of a larger system since Google Books API is not dependent on BookZilla.
    
    % End SubSection
    
    \subsection{Product Functions}
    \label{sub:product_functions}
    % Begin SubSection
    This software is required to perform a variety of Functions. The major functions required include the following.
    \begin{enumerate}[1.]
        \item The Software will search for a list of books based on a distinct and determined set of search terms. For example, a use could use the application to search for books related to the author J.K. Rowling and who's title contains the term Harry or Potter. 
        
        \item The software will search for a book that matches the users description of a book. The software will retrieve a list of books with a story related to the description of the book. The user is also able to search for related books based off of search terms. For example, if the user wanted to search for books with the related terms 'alien', 'space', and 'planet', the software would retrieve a list of related science fiction books.
        
        \item The software will provide the user with a detailed description of the book the user has selected from a retrieved list of books. The user will be able to get in depth information such as Author, Title, Publication Date, Publisher, Description, and ISBNs. 
        
    \end{enumerate}
    % End SubSection
    
    \subsection{User Characteristics}
    \label{sub:user_characteristics}
    % Begin SubSection
    The user must be able use an Android smart-phone. The user must at-least be able to open an App on an Android smart-phone. Since all Android systems are similar, user's ability use one Android smart-phone can be applied across all Android smart-phones.\\\\
    \noindent The user must be familiar with the Android user interface. The user must be able to use basic components such as buttons and text-boxes.\\\\
    \noindent The user must have some knowledge about the book(s) that they are searching for. The user must have enough knowledge of the book(s) to come up with short description(s) of the book(s) that they are searching for. This short description will be used by Bookzilla to find any matching books.
    % End SubSection
    
    \subsection{Constraints}
    \label{sub:constraints}
    % Begin SubSection
    BookZilla must be implemented for the Android operating system. The target platform for BookZilla is the Android operating system.\\\\
    \noindent Final version of BookZilla must be delivered on the April 6\textsuperscript{th}, 2018. On April 6\textsuperscript{th}, 2018, the final version of BookZilla must be demonstrated.\\\\
    BookZilla must follow Google's policies and regulations when using Google Books API. For an example, Google limits the frequency of requests that BookZilla can make to Google Books API. This is done to prevent any misuse of the API. However, it could limit the responsiveness of BookZilla.
    
    \subsection{Assumptions and Dependencies}
    \label{sub:assumptions_and_dependencies}
    % Begin SubSection
    \subsubsection{Dependencies}
    \begin{itemize}
    %	\item List each of the factors that affect the requirements stated in the SRS
        \item Dependencies related to searching for books include author, genre, themes, description, etc. 
        \item Google Books API is a dependency since the operation of BookZilla is reliant on the API.
        \item BookZilla is also dependent on Android operating systems since BookZilla is designed to operate on Android smartphones.
    %	\item These factors are not design constraints on the software but are, rather, any changes to them that can affect the requirements in the SRS
    \subsubsection{Assumptions}
        \item The user must download the application onto their respective android device in order to use the BookZilla application.
        \item The user is familiar with operating an android device.
        \item The android device in which BookZilla is to be deployed on is in working condition.
    \end{itemize}
    % End SubSection
    
    \subsection{Apportioning of Requirements}
    \label{sub:apportioning_of_requirements}
    % Begin SubSection
    \begin{itemize}
        \item The ability of the user to create an account can be developed in version 2.0 or later.
        \item The ability of the user to store previous searches can be available in version 2.0 or later. 
        The ability to export search history will follow from the ability to create an account on BookZilla.
        \item BookZilla version 1.0 will be available only in English. Languages other than English may be available in later versions.
    \end{itemize}
    % End SubSection
    % End Section
    
    \section{Functional Requirements}
    \label{sec:functional_requirements}
    % Begin Section
    
    \begin{enumerate}[{BE}1.]
        \item User desires to \textbf{search for a book}
        \begin{enumerate}[{VP1}.1]
            \item User
                \begin{enumerate}
                    \item User must be able to define Genre of the book in mind.
                    \item  User must be able to define theme of the book in mind.
                    \item  User must be able to define Author or Description of the book in mind.
                \end{enumerate}
        \end{enumerate}
        \item Application outputs answer
        \begin{enumerate}[{VP2}.1]
            \item User
                \begin{enumerate}
                    \item Must be able to look through the outputted by the system
                    \item Must be able to see information about each book in the list such as Title, author, and description or syno
                \end{enumerate}
        \end{enumerate}
    \end{enumerate}
    
    %is this the same as user wants to search for a book?
    
    User wishes to get \textbf{book recommendations} based on author/genre/theme/publisher 
        \begin{enumerate}[{VP1}.1]
            \item User
                \begin{enumerate}
                    \item User must be able to define some viable properties of the book in mind.
                    \begin{itemize}
                        \item Author
                        \item Genre
                        \item Theme
                        \item publisher
                        \item era
                    \end{itemize}
                    \item BookZilla must output a list of books that match the criteria of the user's input.
                \end{enumerate}
        \end{enumerate}
    
    % End Section
    
    \section{Non-Functional Requirements}
    \label{sec:non-functional_requirements}
    % Begin Section
    \subsection{Look and Feel Requirements}
    \label{sub:look_and_feel_requirements}
    % Begin SubSection
    
    \subsubsection{Appearance Requirements}
    \label{ssub:appearance_requirements}
    % Begin SubSubSection
    \begin{enumerate}[{LF}1. ]
        \item The user-interface (\textbf{UI}) must be intuitive.
        \item The UI must be minimal.
        \item The UI must not appear cluttered.
    \end{enumerate}
    % End SubSubSection
    
    \subsubsection{Style Requirements}
    \label{ssub:style_requirements}
    % Begin SubSubSection
    \begin{enumerate}[{LF}1. ]
        \item The UI must be modern.
        \item The UI must be professional.
        \item \textbf{UI elements} must be grouped based on what product function they serve.
    \end{enumerate}
    % End SubSubSection
    
    % End SubSection
    
    \subsection{Usability and Humanity Requirements}
    \label{sub:usability_and_humanity_requirements}
    % Begin SubSection
    
    \subsubsection{Ease of Use Requirements}
    \label{ssub:ease_of_use_requirements}
    % Begin SubSubSection
    \begin{enumerate}[{UH}1. ]
        \item Every product function must not be more than 3 clicks away from the home screen.
        \item The text on screen must be large enough to be easily read.
        \item The UI elements must be large enough to be easily used.
        \item The UI must only use standard UI elements such as text-boxes, buttons.
        \item The UI must provide visual cues as to which UI elements belong to the same product function. For an example, a text-box and a button that are part of the same product function are grouped together and placed next to each other.
    \end{enumerate}
    % End SubSubSection
    
    \subsubsection{Personalization and Internationalization Requirements}
    \label{ssub:personalization_and_internationalization_requirements}
    % Begin SubSubSection
    N/A
    % End SubSubSection
    
    \subsubsection{Learning Requirements}
    \label{ssub:learning_requirements}
    % Begin SubSubSection
    \begin{enumerate}[{UH}1. ]
        \item The program must provide a built-in help function. This help function provides the user with short descriptions of product functions and how to use them.
        \item All UI elements must be paired with a textual label that gives a very short description (one or two words on average) of the the purpose of the corresponding element.
    \end{enumerate}
    % End SubSubSection
    
    \subsubsection{Understandability and Politeness Requirements}
    \label{ssub:understandability_and_politeness_requirements}
    % Begin SubSubSection
    N/A
    % End SubSubSection
    
    \subsubsection{Accessibility Requirements}
    \label{ssub:accessibility_requirements}
    % Begin SubSubSection
    N/A
    % End SubSubSection
    
    % End SubSection
    
    \subsection{Performance Requirements}
    \label{sub:performance_requirements}
    % Begin SubSection
    
    \subsubsection{Speed and Latency Requirements}
    \label{ssub:speed_and_latency_requirements}
    % Begin SubSubSection
    \begin{enumerate}[{PR-SL}1. ]
        \item The application must return a list of candidate books within 5 seconds.
        \item The application must travel from launch to the search page within 2 seconds.
    \end{enumerate}
    % End SubSubSection
    
    \subsubsection{Safety-Critical Requirements}
    \label{ssub:safety_critical_requirements}
    % Begin SubSubSection
    \begin{enumerate}[{PR-SC}1. ]
        \item N/A
    \end{enumerate}
    % End SubSubSection
    
    \subsubsection{Precision or Accuracy Requirements}
    \label{ssub:precision_or_accuracy_requirements}
    % Begin SubSubSection
    \begin{enumerate}[{PR-PA}1. ]
        \item When 3 or more parameters are provided the application must provide the correct result with a minimum 65\% success rate.
        \item The application must conclude the correct genre from related search parameters in at least 70\% of cases.  
    \end{enumerate}
    % End SubSubSection
    
    \subsubsection{Reliability and Availability Requirements}
    \label{ssub:reliability_and_availability_requirements}
    % Begin SubSubSection
    \begin{enumerate}[{PR-RA}1. ]
        \item In the event of an internet interruption, the application must retain the previously inputted information.
        \item The application shall be available for use at all times, provided a suitable internet connection.
        \item The Android version in use must support over 65\% of all current Android devices.
    \end{enumerate}
    % End SubSubSection
    
    \subsubsection{Robustness or Fault-Tolerance Requirements}
    \label{ssub:robustness_or_fault_tolerance_requirements}
    % Begin SubSubSection
    \begin{enumerate}[{PR-FT}1. ]
        \item The application must inform the user when connection is interrupted.
        \item The application must notify the user in the event of an inaccessible API or database.
    \end{enumerate}
    % End SubSubSection
    
    \subsubsection{Capacity Requirements}
    \label{ssub:capacity_requirements}
    % Begin SubSubSection
    \begin{enumerate}[{PR-C}1. ]
        \item N/A
    \end{enumerate}
    % End SubSubSection
    
    \subsubsection{Scalability or Extensibility Requirements}
    \label{ssub:scalability_or_extensibility_requirements}
    % Begin SubSubSection
    \begin{enumerate}[{PR-SE}1. ]
        \item The application must use relative size values on the UI, supporting varying device sizes.
        \item The application must support concurrent identification requests within the desired response times.
    \end{enumerate}
    % End SubSubSection
    
    \subsubsection{Longevity Requirements}
    \label{ssub:longevity_requirements}
    % Begin SubSubSection
    \begin{enumerate}[{PR-L}1. ]
        \item The application shall update all deprecated methods following a new Android version.
    \end{enumerate}
    % End SubSubSection
    
    % End SubSection
    
    \subsection{Operational and Environmental Requirements}
    \label{sub:operational_and_environmental_requirements}
    % Begin SubSection
    
    \subsubsection{Expected Physical Environment}
    \label{ssub:expected_physical_environment}
    % Begin SubSubSection
    \begin{enumerate}[{OE-EPE}1. ]
        \item  The product shall be used in environments where there is an internet connection for mobile devices.
        \item The user is expected to be in a environment where they can access their mobile devices safely.
    \end{enumerate}
    % End SubSubSection
    
    \subsubsection{Requirements for Interfacing with Adjacent Systems}
    \label{ssub:requirements_for_interfacing_with_adjacent_systems}
    % Begin SubSubSection
    \begin{enumerate}[{OE-IAS}1. ]
        \item  NA
    \end{enumerate}
    % End SubSubSection
    
    \subsubsection{Productization Requirements}
    \label{ssub:productization_requirements}
    % Begin SubSubSection
    \begin{enumerate}[{OE-P}1. ]
        \item N/A
    \end{enumerate}
    % End SubSubSection
    
    \subsubsection{Release Requirements}
    \label{ssub:release_requirements}
    % Begin SubSubSection
    \begin{enumerate}[{OE-R}1. ]
        \item The product shall be released free of any major-function related bugs. 
        \item The product shall be available for all Android based devices after Android 5.0.
    \end{enumerate}
    % End SubSubSection
    
    % End SubSection
    
    \subsection{Maintainability and Support Requirements}
    \label{sub:maintainability_and_support_requirements}
    % Begin SubSection
    
    \subsubsection{Maintenance Requirements}
    \label{ssub:maintenance_requirements}
    % Begin SubSubSection
    \begin{enumerate}[{MS-M}1. ]
        \item The project shall be maintained regularly as updates appear for used APIs.
    \end{enumerate}
    % End SubSubSection
    
    \subsubsection{Supportability Requirements}
    \label{ssub:supportability_requirements}
    % Begin SubSubSection
    \begin{enumerate}[{MS-S}1. ]
        \item The product shall be accessible from any Android device running Android 5.0 or greater.
    \end{enumerate}
    % End SubSubSection
    
    \subsubsection{Adaptability Requirements}
    \label{ssub:adaptability_requirements}
    % Begin SubSubSection
    \begin{enumerate}[{MS-A}1. ]
        \item The product should be easily modifiable in the event that new features need to be added. 
    
    \end{enumerate}
    % End SubSubSection
    
    % End SubSection
    
    \subsection{Security Requirements}
    \label{sub:security_requirements}
    % Begin SubSection
    
    \subsubsection{Access Requirements}
    \label{ssub:access_requirements}
    % Begin SubSubSection
    \begin{enumerate}[{SR-A}1. ]
        \item Users must not be able to access to API keys used in the product.
    \end{enumerate}
    % End SubSubSection
    
    \subsubsection{Integrity Requirements}
    \label{ssub:integrity_requirements}
    % Begin SubSubSection
    \begin{enumerate}[{SR-I}1. ]
        \item N/A
    \end{enumerate}
    % End SubSubSection
    
    \subsubsection{Privacy Requirements}
    \label{ssub:privacy_requirements}
    % Begin SubSubSection
    \begin{enumerate}[{SR-P}1. ]
        \item User data will not be retained or collected.
        \item The application shall not allow unauthorized users or programs to access any stored data.
    \end{enumerate}
    % End SubSubSection
    
    \subsubsection{Audit Requirements}
    \label{ssub:audit_requirements}
    % Begin SubSubSection
    \begin{enumerate}[{SR-AU}1. ]
        \item N/A
    \end{enumerate}
    % End SubSubSection
    
    \subsubsection{Immunity Requirements}
    \label{ssub:immunity_requirements}
    % Begin SubSubSection
    \begin{enumerate}[{SR-I}1. ]
        \item N/A
    \end{enumerate}
    % End SubSubSection
    
    % End SubSection
    
    \subsection{Cultural and Political Requirements}
    \label{sub:cultural_and_political_requirements}
    % Begin SubSection
    
    \subsubsection{Cultural Requirements}
    \label{ssub:cultural_requirements}
    % Begin SubSubSection
    \begin{enumerate}[{CP-C}1. ]
        \item The product shall use American spelling.
    \end{enumerate}
    % End SubSubSection
    
    \subsubsection{Political Requirements}
    \label{ssub:political_requirements}
    % Begin SubSubSection
    \begin{enumerate}[{CP-P}1. ]
        \item The application must not be offensive in it's selection of search results unless the context of the search can be explicitly deemed as offensive, in which case it will display the requested content. 
        \item The application shall only display content related to the context of the search properties provided by the user. 
    \end{enumerate}
    % End SubSubSection
    
    % End SubSection
    
    \subsection{Legal Requirements}
    \label{sub:legal_requirements}
    % Begin SubSection
    
    \subsubsection{Compliance Requirements}
    \label{ssub:compliance_requirements}
    % Begin SubSubSection
    \begin{enumerate}[{LR-C}1. ]
        \item Bookzilla must adhere to the guidelines laid out in the Google Books API terms of service.
        
    \end{enumerate}
    % End SubSubSection
    
    \subsubsection{Standards Requirements}
    \label{ssub:standards_requirements}
    % Begin SubSubSection
    \begin{enumerate}[{LR-S}1. ]
        \item N/A
    \end{enumerate}
    % End SubSubSection
    
    % End SubSection
    
    % End Section
    \newpage
    \appendix
    \section{Division of Labour}
    \label{sec:division_of_labour}
    % Begin Section
    \begin{table}[H]
        \begin{center}
            \rowcolors{1}{white}{light-gray}
            \begin{tabular}{| p{0.30\linewidth} | p{0.70\linewidth} |}
                \hline
                \textbf{Member} & \textbf{Contributions}\\
                \hline
                Daniel & Sections 2.1, 2.2, 4.4, 4.5, 4.6, 4.7, 4.8\\
                
                Razan & Sections 2.5, 2.6, 3\\
                
                Sachin & Sections 2.3, 2.4, 4.1, 4.2\\
                
                Michael & Sections 1.3, 1.4, 1.5, 4.3\\
                
                Mohamed & Sections 1.1, 1.2, 3\\
                \hline
            \end{tabular}
            \caption{Contributions} \label{tab:Contributions}
        \end{center}
    \end{table}
    % End Section
    \end{document}
    